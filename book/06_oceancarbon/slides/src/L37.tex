\documentclass[aspectratio=169]{beamer}
% \documentclass[aspectratio=169, handout]{beamer}

%---------------------------
%       Beamer Cheat Sheet
%---------------------------
% https://www.cpt.univ-mrs.fr/~masson/latex/Beamer-appearance-cheat-sheet.pdf

%---------------------------
%       Set Theme and Colors
%---------------------------

\usetheme[width=.1\paperwidth]{Hannover}
% \setbeamertemplate{sidebar canvas right}[vertical shading][top=red,bottom=blue]

\definecolor{QESblue}{HTML}{8AD2ED}
\definecolor{QESdarkblue}{HTML}{187ca2}
\definecolor{QESlightblue}{HTML}{c4e8f6}

\setbeamercolor{sidebar}{bg=QESlightblue}
\setbeamercolor{titlelike}{fg=QESdarkblue}

\setbeamercolor{palette sidebar secondary}{fg=QESdarkblue}
\setbeamercolor{title in sidebar}{fg=QESdarkblue}
\setbeamercolor{author in sidebar}{fg=QESdarkblue}

\setbeamercolor{section in sidebar}{fg=QESdarkblue}
\setbeamercolor{subsection in sidebar}{fg=QESdarkblue}

\addtobeamertemplate{sidebar left}{}{\vfill \hspace{.006\paperwidth} \includegraphics[width=.08\paperwidth]{../../../logo.png} \vspace{.006\paperwidth} } 
% \addtobeamertemplate{sidebar left}{}{\vfill \hspace{.00001\paperwidth} \includegraphics[width=.093\paperwidth]{../../figures/qes-qr.png} \vspace{.003\paperwidth} } 

%---------------------------
%       No navigation symbols
%---------------------------
\setbeamertemplate{navigation symbols}{}

%---------------------------
%       Set Fonts
%---------------------------
\usepackage{helvet}
\renewcommand{\familydefault}{\sfdefault}
\usepackage{sansmathfonts}
\usepackage{upgreek}

\setbeamerfont{frametitle}{series=\bfseries, size=\Large}
\setbeamerfont{title in sidebar}{series=\bfseries, size=\small}
\setbeamertemplate{caption}{\it\raggedright\insertcaption\par}

%---------------------------
%       Math font packages
%---------------------------
\usepackage{dsfont, amsmath, amsthm, mathtools}
\usepackage{bbm, bm}
\usepackage[T1]{fontenc}
\usepackage[version=3]{mhchem}

%---------------------------
%       Figure packages
%---------------------------
\usepackage{graphicx}
\graphicspath{{../../figures}}

\usepackage{epstopdf}
\usepackage{color}

\setbeamerfont{caption}{size=\footnotesize}

\usepackage{subfigure}

%---------------------------
%       Manual placement packages
%---------------------------
\usepackage{tikz}
\usetikzlibrary{calc}

%---------------------------
%       Local Macros
%---------------------------

\newcommand{\manualpic}[4]{
    % inputs {filename}{figure options}{x offset}{y offset}
    \tikz[remember picture, overlay] \node[anchor=center] at ($(current page.center)+(#3,#4)$) {\includegraphics[#2]{#1}};
}

\newcommand{\manualtext}[3]{
    % inputs {text}{x offset}{y offset}
    \tikz[remember picture, overlay] \node[anchor=center] at ($(current page.center)+(#2,#3)$) {#1};
}

\newcommand{\manualtextleft}[3]{
    % inputs {text}{x offset}{y offset}
    \tikz[remember picture, overlay] \node[anchor=west] at ($(current page.center)+(#2,#3)$) {#1};
}

\newcommand{\manualtextright}[3]{
    % inputs {text}{x offset}{y offset}
    \tikz[remember picture, overlay] \node[anchor=east] at ($(current page.center)+(#2,#3)$) {#1};
}

\newcommand{\slidereference}[1]{
    \manualtextleft{{\color{QESdarkblue}\tiny #1}}{-0.47\linewidth}{-0.47\textheight}
}

\newcommand{\dv}[2]{\frac{\mathrm{d}#1}{\mathrm{d}#2}}

\title{Ocean Carbon}
\author{3/5}

\begin{document}

\begin{frame}{The Biological Pump}
    
    \includegraphics[width=\linewidth, totalheight=0.8\textheight, keepaspectratio]{ocean-3box-CO2-bio.png}
        
\end{frame}

\begin{frame}<handout:0>{Patterns of Ocean Carbon}
    \centering
    \slidereference{Sarmiento \& Gruber (2006)}

    \includegraphics<1>[width=\linewidth, totalheight=0.75\textheight, keepaspectratio]{carbon-ocean-atmos.png}

    \includegraphics<2>[width=\linewidth, totalheight=0.75\textheight, keepaspectratio]{carbon-cx-dic.png}
    \only<2>{
        \manualtext{\rotatebox{90}{$\mathrm{\upmu mol~kg^{-1}}$}}{0.52\linewidth}{-0.6}
    }

\end{frame}

\section{Organic Matter}

\begin{frame}{Phytoplankton}
    \centering

    \includegraphics<1|handout:1>[width=\linewidth, totalheight=0.75\textheight, keepaspectratio]{seawifs-chlorophyll.png}
    \only<1>{\slidereference{SeaWIFS, NASA}}

    \only<2|handout:2>{
    $$
    \mathrm{\underbrace{6 CO_2}_{carbon~dioxide} + \underbrace{6 H_2O}_{water} \xrightarrow[+light]{photosynthesis} \underbrace{C_6H_{12}O_6}_{glucose} + \underbrace{6 O_2}_{oxygen}}
    $$
    }

\end{frame}

\begin{frame}{Photosynthesis and pH?}
    
    \centering

    \onslide<1-|handout:0>{
        $$
        \ce{CO2 + H2O} \rightleftharpoons \ce{HCO3- + H+} \rightleftharpoons \ce{CO3{2-} + H+}
        $$
    }

    \onslide<2|handout:0>{
        \includegraphics[width=\linewidth, totalheight=0.65\textheight, keepaspectratio]{carbon-bjerrum.png}
    }

\end{frame}

\begin{frame}{Plankton}
    \centering

    \only<1|handout:1>{\slidereference{Annegret Stuhr, GEOMAR}}
    \includegraphics<1|handout:1>[width=\linewidth, totalheight=0.75\textheight, keepaspectratio]{carbon-plankton.jpg}

    \only<2|handout:2>{\slidereference{@PlanktonPundit}}
    \includegraphics<2|handout:2>[width=\linewidth, totalheight=0.75\textheight, keepaspectratio]{carbon-plankton-foodweb.jpeg}

\end{frame}

\begin{frame}{Patterns of Productivity}
    \centering

    \includegraphics[width=\linewidth, totalheight=0.75\textheight, keepaspectratio]{carbon-npp.png}

\end{frame}

\begin{frame}{Organic Matter}

    \centering
    Organic matter is more than glucose - the organism requires nutrients to grow:

    \begin{align*}
    \mathrm{\underbrace{106 CO_2}_{carbon~dioxide} + \underbrace{16 NO_3 + H_3PO_4}_{nutrients} + 78 H_2O} &+ \mathrm{18H^+} & \\
    & \mathrm{ \xrightarrow[+light]{photosynthesis}} & \\
    && \mathrm{\underbrace{C_{106}H_{175}O_{42}N_{16}P}_{organic~matter} + \underbrace{150 O_2}_{oxygen}} \\
    \end{align*}

    \onslide<2|handout:1>{The main limiting nutrients are nitrate (\ce{NO3}) and phosphate \ce{PO4}.}

\end{frame}
 
\section{Nutrients}

\begin{frame}{Nutrient Requirements}
    \begin{columns}
        \begin{column}{0.4\linewidth}
            \includegraphics[width=\linewidth, totalheight=0.75\textheight, keepaspectratio]{carbon-redfield.png}
        \end{column}
        \begin{column}{0.5\linewidth}
            \begin{itemize}
                \item Organic matter contains a relatively constant ratio of C:N:P at 106:16:1.
                \item This is known as the \textbf{Redfield Ratio}.
                \item There is a tight coupling between the N:P ratio dissolved in the ocean and in organic matter.
            \end{itemize}
        \end{column}
    \end{columns}
\end{frame}

\begin{frame}{Nutrient Sources}
    \begin{itemize}
        \item \textbf{Phosphate} is derived from the dissolution of rocks on the continents. Supplied either via rivers, or by wind-blown dust from deserts.
        \item \textbf{Nitrate} also comes from terrestrial sources, but can also be captured from the air (\ce{N2}, 79\% of atmosphere) by specialised `nitrogen-fixing' phytoplankton.
    \end{itemize} 
\end{frame}

\begin{frame}{Nutrient Excess?}
    \centering
    \includegraphics[width=\linewidth, totalheight=0.75\textheight, keepaspectratio]{ocean-surface-po4.png}
\end{frame}

\begin{frame}{Micronutrients}
    \centering
    \includegraphics[width=\linewidth, totalheight=0.75\textheight, keepaspectratio]{carbon-nutrient-limitation.png}
\end{frame}

\begin{frame}{Nutrient Limitation}

    \begin{itemize}
        \item The availability of N, P limits phytoplankton growth.
        \item In some regions, micronutrients are also critical.
        \item For our model, we will use \ce{PO4} to limit phytoplankton growth.
    \end{itemize}

\end{frame}

\section{Light}

\begin{frame}{Light Limitation}
    \begin{columns}
        \begin{column}{0.7\linewidth}
            \includegraphics<2|handout:1>[width=\linewidth, totalheight=\textheight, keepaspectratio]{carbon-biopump-light.png}
        \end{column}
        \begin{column}{0.3\linewidth}
            \includegraphics<1|handout:1>[width=\linewidth, totalheight=\textheight, keepaspectratio]{carbon-light-limitation.png}
        \end{column}
    \end{columns}
\end{frame}

\begin{frame}{Compensation Depth}
    \centering

    \includegraphics[width=\linewidth, totalheight=0.75\textheight, keepaspectratio]{carbon-compensation-depth.png}
\end{frame}

\begin{frame}{Approximating Light Limitation}
    \begin{itemize}
        \item The high latitudes are dark with a deep mixed layer depth (MLD) for part of the year.
        \item Nutrients accumulate in the winter months, when phytoplankton cannot grow.
        \item Nutrients are removed less quickly from the surface, so have a longer lifetime ($\tau$) in high latitudes.
    \end{itemize}
\end{frame}

\section{Biological Pump}

\begin{frame}{Marine Snow}

    \only<1|handout:0>{\slidereference{@PlanktonPundit}}
    \includegraphics<1|handout:0>[width=\linewidth, totalheight=0.75\textheight, keepaspectratio]{carbon-plankton-foodweb.jpeg}

    \only<2|handout:1>{\slidereference{Sarmiento \& Gruber (2006)}}
    \includegraphics<2|handout:1>[width=\linewidth, totalheight=0.75\textheight, keepaspectratio]{carbon-npp.png}

    \only<3|handout:2>{
        \slidereference{Omand et al. (2020)}

        \begin{columns}
            \begin{column}{0.3\linewidth}
                \includegraphics[width=\linewidth, totalheight=0.9\textheight, keepaspectratio]{carbon-particle-sinking.png}
            \end{column}
            \begin{column}{0.6\linewidth}
                \begin{itemize}
                    \item Organisms die and defecate, producing ``marine snow'' particles and aggregates.
                    \item Particles sink through the water column, and are `remineralised' by bacteria.
                    \item Remineralisation dissolves the particles back into the water column.
                    \item $\sim$80\% is remineralised in the surface ocean, $\sim$20\% is exported to the deep ocean.
                \end{itemize}
            \end{column}
        \end{columns}
    }

    \only<4|handout:3>{\slidereference{Sarmiento \& Gruber (2006)}}
    \includegraphics<4|handout:3>[width=\linewidth, totalheight=0.75\textheight, keepaspectratio]{carbon-poc-export.png}

\end{frame}

\begin{frame}{The Biological Pump}

    \centering

    \includegraphics<1|handout:0>[width=\linewidth, totalheight=0.65\textheight, keepaspectratio]{carbon-circ-6-biopump-const.png}

    \includegraphics<2|handout:0>[width=\linewidth, totalheight=0.65\textheight, keepaspectratio]{carbon-circ-7-biopump-deep.png}

    \includegraphics<3|handout:0>[width=\linewidth, totalheight=0.65\textheight, keepaspectratio]{carbon-circ-8-biopump-surf.png}

    \includegraphics<4|handout:1>[width=\linewidth, totalheight=0.65\textheight, keepaspectratio]{carbon-circ-9-biopump-full.png}

    \includegraphics<5|handout:2>[width=\linewidth, totalheight=0.75\textheight, keepaspectratio]{carbon-cx-po4.png}

    \includegraphics<6|handout:0>[width=\linewidth, totalheight=0.75\textheight, keepaspectratio]{carbon-cx-dic.png}

    \includegraphics<7|handout:3>[width=\linewidth, totalheight=0.75\textheight, keepaspectratio]{carbon-Csoft.png}


\end{frame}

\section{Modelling Ocean Carbon}

\begin{frame}{Modelling the Biological Pump}

    \begin{enumerate}
        \item Add a nutrient to limit biological productivity (\ce{PO4})
        \item Parameterise the link between photosynthesis, remineralisation and carbon chemistry.
    \end{enumerate}

\end{frame}

\begin{frame}{1. Adding a Nutrient}

    \ce{PO4} is onsumed by phytoplankton and remineralised at depth. 
    
    Describe uptake and export with $\tau^P$, which will be longer in high latitudes because of light limitation. Note that $\tau^P$ is \textit{not} production, but \textit{export} - how much of the organic matter produced by phytoplankton leaves the surface and enters the deep ocean.

    \begin{align*}
        \mathrm{\dv{P_L}{t}} &= \mathrm{[\mathrm{transport}] - \frac{1}{\tau^P_L} P_L} \\
        \mathrm{\dv{P_H}{t}} &= \mathrm{[\mathrm{transport}] - \frac{1}{\tau^P_H} P_H} \\
        \mathrm{\dv{P_D}{t}} &= \mathrm{[\mathrm{transport}] + \left. \left( \frac{V_L}{\tau^P_L} P_L + \frac{V_L}{\tau^P_H} P_H \right) \middle/ V_D \right.}
    \end{align*}

\end{frame}

\begin{frame}{2. Coupling productivity and Carbon}

    Photosynthesis and remineralisation affect both DIC and TA, and can be simplified to:

    $$
    \mathrm{P + 106 DIC - 18 TA \xrightleftharpoons[remineralisation]{photosynthesis} [organic~matter]}
    $$

    We can link DIC and TA to PO4 via the Redfield Ratio as:

    \begin{align*}
        \mathrm{\frac{d[DIC]_{bio}}{dt}} &= \mathrm{106 \frac{dP_{bio}}{dt}} \\
        \mathrm{\frac{d[TA]_{bio}}{dt}} &= \mathrm{-18 \frac{dP_{bio}}{dt}}
    \end{align*}

    Which we include alongside transport and atmospheric echange\dots

\end{frame}

\begin{frame}{2. Coupling productivity and Carbon: DIC}

    \begin{align*}
        \mathrm{\dv{DIC_L}{t}} &= \mathrm{\begin{bmatrix} \mathrm{transport} \\ \mathrm{CO_2~exch.}\end{bmatrix} - 106 \frac{1}{\tau^P_L} P_L} \\
        \mathrm{\dv{DIC_H}{t}} &= \mathrm{\begin{bmatrix} \mathrm{transport} \\ \mathrm{CO_2~exch.}\end{bmatrix} - 106 \frac{1}{\tau^P_H} P_H} \\
        \mathrm{\dv{DIC_D}{t}} &= \mathrm{[\mathrm{transport}] + \left. \left( 106 \frac{V_L}{\tau^P_L} P_L + 106 \frac{V_H}{\tau^P_H} P_H \right) \middle / V_D \right.}
    \end{align*}

\end{frame}

\begin{frame}{2. Coupling productivity and Carbon: TA}
    
    \begin{align*}
        \mathrm{\dv{TA_L}{t}} &= \mathrm{[\mathrm{transport}] + 18 \frac{1}{\tau^P_L} P_L} \\
        \mathrm{\dv{TA_H}{t}} &= \mathrm{[\mathrm{transport}] + 18 \frac{1}{\tau^P_H} P_H} \\
        \mathrm{\dv{TA_D}{t}} &= \mathrm{[\mathrm{transport}] - \left. \left( 18 \frac{V_L}{\tau^P_L} P_L + 18 \frac{V_H}{\tau^P_H} P_H \right) \middle / V_D \right.}
    \end{align*}
    
    \end{frame}

\begin{frame}{Impact of Biological Pump}
    \centering

    \includegraphics<1|handout:1>[width=\linewidth, totalheight=0.8\textheight, keepaspectratio]{carbon-model-DIC-TA-pCO2.png}
    \only<1|handout:1>{$\mathrm{pCO_2} \sim 1300$}

    \includegraphics<2|handout:2>[width=\linewidth, totalheight=0.8\textheight, keepaspectratio]{carbon-model-DIC-TA-pCO2-bio.png}
    \only<2|handout:2>{$\mathrm{pCO_2} \sim 350$}
    
\end{frame}

\end{document}



% \begin{frame}{TITLE}
% \end{frame}